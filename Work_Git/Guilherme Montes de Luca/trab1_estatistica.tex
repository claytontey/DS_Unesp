% Clayton Pereira, 23/04/2025
% Louvado seja o Senhor meu DEUS.


\documentclass[12pt,a4paper]{article}
\usepackage[utf8]{inputenc}
\usepackage[brazil]{babel}
\usepackage{graphicx}
\usepackage{fancyhdr}
\usepackage{geometry}
\usepackage{parskip}
\usepackage{amsmath}
\usepackage{caption}
\usepackage{float}
\usepackage{hyperref}

\geometry{margin=2.5cm}

%\pagestyle{fancy}

% Cabeçalho para as páginas seguintes
%\fancyhead[L]{\includegraphics[width=2.5cm]%{figs/aula2/unesp-logo.jpg}} 
%\fancyhead[C]{}
%\fancyhead[R]{\textbf{Relatório de Estatística Aplicada}}
%\fancyfoot[C]{\thepage}

\begin{document}


\hfill
\vspace{1cm}

\textbf{Nome: Guilherme Montes de Luca} 

\textbf{Curso/Turma: Ciência da Computação} 

\textbf{Data de Entrega: 07/05/2025}

\vspace{1cm}

\section*{1. Tema da Pesquisa}
O tema Sono x Produtividade investiga a relação entre os padrões de sono e a sensação de produtividade ao longo do dia. A pesquisa se concentra em entender como a quantidade e a qualidade do sono influenciam o rendimento diário. Isso pode ser analisado a partir de dados sobre as horas dormidas, os horários de dormir e acordar, e a percepção de energia ao acordar. Através dessa análise, busca-se identificar como o sono afeta a disposição e a capacidade de realizar tarefas, tanto cognitivas quanto físicas, durante o dia. O objetivo é compreender melhor como o sono impacta a produtividade e como ajustes nos hábitos de sono podem melhorar o desempenho diário.

\section*{2. Objetivo}
Com esse formulário, o objetivo é investigar a relação entre os hábitos de sono e a percepção de produtividade das pessoas ao longo do dia. Através das perguntas, busca-se entender quantas horas de sono são comuns, como o indivíduo se sente ao acordar e durante o dia, além de identificar padrões específicos que podem estar relacionados à energia, cansaço e produtividade. Também é possível avaliar como dificuldades na manutenção de uma boa rotina de sono, como o uso de eletrônicos, estresse ou distúrbios do sono, impactam a qualidade do sono e a capacidade de concentração ou desempenho no trabalho ou estudos. Além disso, o formulário busca descobrir se as pessoas tentam adotar estratégias para melhorar o sono e se essas tentativas influenciam de fato na produtividade. O intuito é fornecer insights sobre como ajustar os hábitos de sono pode melhorar o rendimento diário, considerando fatores como o horário de sono, os recursos usados para melhorar o descanso e a percepção geral de produtividade.


\section*{3. Estrutura do Formulário}
\begin{enumerate}
    \item Quantas horas você costuma dormir por noite, em média? \\
    \textbf{Tipo}: Qualitativa Ordinal
    \item Você se sente descansado ao acordar? \\
    \textbf{Tipo}: Qualitativa Ordinal
    \item Com que frequência você sente sono ou cansaço durante o dia? \\
    \textbf{Tipo}: Qualitativa Ordinal
    \item Em qual período do dia você se sente mais produtivo? \\
    \textbf{Tipo}: Qualitativa Nominal
    \item Você percebe uma relação entre a qualidade do seu sono e sua produtividade? \\
    \textbf{Tipo}: Qualitativa Ordinal
    \item Qual a sua principal dificuldade em manter uma boa rotina de sono? \\
    \textbf{Tipo}: Qualitativa Nominal
    \item Você utiliza algum recurso para melhorar o sono? \\
    \textbf{Tipo}: Qualitativa Nominal
    \item Como você avalia sua produtividade geral durante o dia? \\
    \textbf{Tipo}: Qualitativa Ordinal
    \item Você já tentou ajustar seus hábitos de sono para melhorar a produtividade? \\
    \textbf{Tipo}: Qualitativa Nominal
    \item Com que frequência você dorme menos do que gostaria por causa de compromissos ou prazos? \\
    \textbf{Tipo}: Qualitativa Ordinal
\end{enumerate}


\section*{4. Coleta de Dados}
Foram obtidas 30 respostas por meio de um formulário online, distribuído através do Google Forms. As respostas foram coletadas principalmente de amigos e familiares, que participaram da pesquisa com o intuito de compartilhar suas experiências pessoais relacionadas aos hábitos de sono e produtividade. O perfil dos participantes, portanto, é composto por pessoas próximas ao pesquisador, com diferentes rotinas e estilos de vida, mas todas com uma conexão direta com o autor do estudo.


\section*{5. Análise Estatística}

A maioria das pessoas (moda e mediana) relata dormir 7 horas por noite, o que está próximo da recomendação ideal. No entanto, a média mais baixa (6,20) sugere que parte dos respondentes dorme bem menos, puxando a média para baixo. Isso indica uma distribuição levemente assimétrica à esquerda, com alguns casos de sono insuficiente.

Apesar de uma moda relativamente alta (4,00), a média de 2,63 revela que muitos indivíduos não se sentem totalmente descansados ao acordar. A diferença entre a moda e a média pode sugerir que uma parte considerável das respostas está concentrada nas notas mais baixas, resultando em uma percepção geral de sono não reparador.

A maior parte dos respondentes (11 pessoas) dorme entre 6 e 8 horas, alinhando-se às recomendações de sono saudável. No entanto, há um número relevante de pessoas (7) que dormem menos de 4 horas, o que pode comprometer a saúde e o desempenho diário. A distribuição indica uma predominância nessa faixa intermediária, mas ainda com presença significativa de extremos.

Apesar de uma boa parte relatar que sempre se sente descansado (9 respostas), os demais grupos (nunca, raramente, frequentemente) ainda somam uma quantidade expressiva. Isso sugere que, mesmo com uma boa duração de sono em alguns casos, a qualidade do descanso pode não estar ideal para todos.

A noite foi o período mais citado como o mais produtivo (11 pessoas), seguida pela manhã (10). Isso mostra que há variação no ritmo biológico entre os indivíduos, com um leve destaque para cronótipos noturnos. Apenas 4 pessoas se consideram mais produtivas à tarde, e 5 não têm um padrão definido.


\begin{figure}
    \centering
    \includegraphics[width=1\linewidth]{image.png}
    \caption{Horas de Sono por Noite, Sensação de Descanso ao Acordar e Período do Dia Mais Produtivo}
    \label{fig:enter-label}
\end{figure}

\begin{figure}
    \centering
    \includegraphics[width=1\linewidth]{image2.png}
    \caption{Distribuição das Horas de Sono e Período do Dia Mais Produtivo}
    \label{fig:enter-label}
\end{figure}

\section*{6. Conclusão / Insight}
Com base nas 30 respostas obtidas, algumas conclusões importantes podem ser extraídas. Observando as variáveis, percebe-se que a maioria dos participantes dorme entre 6 e 8 horas por noite, seguido por um número significativo que dorme menos de 4 horas, o que pode indicar uma rotina de sono irregular ou com poucos períodos de descanso adequados. A relação entre sono e sensação de produtividade é interessante, com a maioria dos participantes afirmando perceber uma relação positiva entre a qualidade do sono e sua produtividade.

Em relação à sensação de descanso ao acordar, muitos participantes indicam que se sentem descansados com uma frequência variada, mas um número considerável aponta que se sentem frequentemente cansados, o que pode ser uma consequência de uma quantidade insuficiente de sono ou de distúrbios no sono.

Quanto à frequência de cansaço durante o dia, a maioria dos participantes relata sentir sono ou cansaço com regularidade, o que sugere que a qualidade do sono tem um impacto direto na energia e no rendimento diário. A análise também mostra que o período de maior produtividade tende a ser na manhã ou na noite, com algumas pessoas mencionando um padrão irregular, sugerindo que o relógio biológico e a adaptação individual desempenham papéis importantes.

No que diz respeito às dificuldades em manter uma boa rotina de sono, o uso excessivo de eletrônicos antes de dormir e o estresse ou ansiedade são as principais causas relatadas pelos participantes. Isso reforça a ideia de que fatores externos, como o uso de tecnologia, impactam diretamente a qualidade do sono.

Para melhorar a qualidade do sono, muitos utilizam recursos como meditação, leitura, ou o uso de aplicativos para relaxamento. No entanto, a eficácia dessas estratégias parece ser variada, já que nem todos relatam melhorias significativas na produtividade após ajustar seus hábitos de sono. Alguns participantes indicam que tentaram melhorar seus hábitos, mas sem sucesso, enquanto outros acreditam que ajustes como o uso de medicamentos ou mudanças no ambiente de sono trouxeram resultados positivos.

Em resumo, os dados mostram que, embora haja um reconhecimento da importância do sono para a produtividade, muitos participantes ainda enfrentam dificuldades em manter uma boa rotina de sono, com implicações diretas em seu desempenho diário. A análise revela um padrão comum de dificuldades relacionadas ao estresse, uso de tecnologia e irregularidades nos horários de sono, que afetam tanto o descanso quanto a produtividade.


\section*{7. Reflexão Final}
Durante a realização da atividade, aprendeu-se que muitos participantes enfrentam dificuldades em manter uma boa qualidade de sono devido ao uso excessivo de eletrônicos, estresse e horários irregulares. Apesar de perceberem a relação entre sono e produtividade, poucos ajustaram efetivamente seus hábitos para melhorar essa relação. A amostra revelou padrões interessantes, como a falta de sono em muitas pessoas, o que impacta negativamente na produtividade.

Em uma próxima pesquisa, seria útil ampliar a amostra, incluindo pessoas de diferentes idades e contextos, e explorar mais a fundo as causas dos distúrbios do sono. Além disso, seria interessante utilizar abordagens como entrevistas ou grupos focais para obter dados mais ricos e considerar o uso de dispositivos para monitoramento de sono, a fim de melhorar a precisão dos dados coletados.



\hfill
\vspace{1cm}
\section*{Atividade – Estatística Aplicada com Formulário}

Nesta atividade, cada aluno deverá desenvolver um formulário (Google Forms) com pelo menos \textbf{8 perguntas}, aplicando os conhecimentos de:

\begin{itemize}
    \item Medidas de frequência (absoluta, relativa e acumulada);
    \item Estatística descritiva (média, moda, mediana e desvio padrão);
    \item Tipos de variáveis (qualitativa nominal, qualitativa ordinal, quantitativa discreta e contínua).
\end{itemize}

O formulário deve ser aplicado com, no mínimo, \textbf{20 respondentes reais}. Ao final, o aluno deverá apresentar um relatório com análise dos dados, gráficos e medidas calculadas.

\subsection*{Temas sugeridos}

\begin{enumerate}
    \item \textbf{Sono x Produtividade:} Avaliar padrões de sono e sensação de produtividade. Perguntas podem incluir horas dormidas, horário de dormir, nível de energia ao acordar, percepção de rendimento diário etc.
    
    \item \textbf{Tempo de Estudo x Notas:} Levantar hábitos de estudo e percepção de desempenho acadêmico. Ex: horas de estudo, uso de resumos, revisão antes da prova, notas médias recentes.
    
    \item \textbf{Redes Sociais x Ansiedade:} Investigar tempo de uso das redes e sensações associadas. Pode incluir tipo de rede mais usada, tempo médio por dia, sentimentos após uso, percepção de dependência.
    
    \item \textbf{Estresse x Alimentação:} Avaliar frequência de consumo de fast food, refeições em horários irregulares e níveis de estresse percebidos.
    
    \item \textbf{Atividade Física x Qualidade do Sono:} Levantar frequência de prática de exercícios e qualidade do sono (sono leve/profundo, interrupções, insônia etc.).
    
    \item \textbf{Música x Raciocínio:} Identificar se quem toca instrumentos musicais percebe maior concentração ou desempenho lógico. Perguntas sobre prática musical, tipo de música, tempo de prática, preferências etc.
    
    \item \textbf{Treinos x Autoestima:} Analisar relação entre frequência de exercícios físicos e percepção de autoestima. Incluir perguntas sobre motivação, frequência semanal, tipo de treino, satisfação pessoal etc.
    
    \item \textbf{Cafeína x Produtividade:} Estudar consumo de café/energéticos e impacto no foco. Levantar quantidade de consumo, horário de ingestão, percepção de desempenho antes e após consumir.
    
    \item \textbf{Animais x Produção (Agro):} Coletar dados de pequenas propriedades. Ex: número de vacas, tipo de ração, litros de leite/dia, produtividade percebida.

\hfill
\vspace{1cm}
    
    \item \textbf{Lazer x Felicidade:} Medir frequência de atividades de lazer e percepção de felicidade. Perguntas podem incluir tipo de lazer, frequência, tempo médio, avaliação da semana.
\end{enumerate}

\subsection*{Instruções para entrega do relatório}

\begin{itemize}
    \item Apresentar o link do formulário e um resumo das perguntas aplicadas.
    \item Classificar cada pergunta de acordo com o tipo de variável (qualitativa/quantitativa, discreta/contínua, ordinal/nominal).
    \item Apresentar as tabelas de frequência (absoluta, relativa e acumulada) para ao menos 3 perguntas.
    \item Calcular média, moda e mediana de pelo menos 2 variáveis quantitativas.
    \item Incluir ao menos 2 gráficos (barras ou histograma).
    \item Incluir uma reflexão final com possíveis conclusões ou observações.
    \item O código, conjunto de dados e suas análises deverão estar no \href{https://github.com/claytontey/DS_Unesp/tree/main/Work_Git}{WorkGit} em um diretório com seu nome.
\end{itemize}


Bom trabalho!!!

\end{document}
