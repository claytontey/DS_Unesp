% Clayton Pereira, 23/04/2025
% Louvado seja o Senhor meu DEUS.


\documentclass[12pt,a4paper]{article}
\usepackage[utf8]{inputenc}
\usepackage[brazil]{babel}
\usepackage{graphicx}
\usepackage{fancyhdr}
\usepackage{geometry}
\usepackage{parskip}
\usepackage{amsmath}
\usepackage{caption}
\usepackage{float}
\usepackage{hyperref}

\geometry{margin=2.5cm}
\pagestyle{fancy}

% Cabeçalho para as páginas seguintes
\fancyhead[L]{\includegraphics[width=2.5cm]{figs/aula2/unesp-logo.jpg}} 
\fancyhead[C]{}
\fancyhead[R]{\textbf{Relatório de Estatística Aplicada}}
\fancyfoot[C]{\thepage}

\begin{document}


\hfill
\vspace{1cm}

\textbf{Nome:} \rule{10cm}{0.4pt}

\textbf{Curso/Turma:} \rule{8cm}{0.4pt}

\textbf{Data de Entrega:} \rule{6cm}{0.4pt}

\vspace{1cm}

\section*{1. Tema da Pesquisa}
Descreva brevemente o tema escolhido e a motivação para a escolha.

\section*{2. Objetivo}
Explique o que você deseja investigar com o formulário.

\section*{3. Estrutura do Formulário}
Liste ou resuma as perguntas incluídas. Identifique os tipos de variáveis (quantitativas, qualitativas, múltipla escolha etc).

\section*{4. Coleta de Dados}
Informe quantas respostas foram obtidas, o meio de divulgação e o perfil dos participantes.

\section*{5. Análise Estatística}
Apresente gráficos e tabelas com análise dos dados. Utilize medidas como média, mediana, moda, proporções, etc.

\section*{6. Conclusão / Insight}
Explique quais conclusões foram tiradas da análise. Há algum padrão ou comportamento interessante?

\section*{7. Reflexão Final}
O que foi aprendido durante a realização da atividade? O que faria diferente em uma próxima pesquisa?

\newpage
\hfill
\vspace{1cm}
\section*{Atividade – Estatística Aplicada com Formulário}

Nesta atividade, cada aluno deverá desenvolver um formulário (Google Forms) com pelo menos \textbf{8 perguntas}, aplicando os conhecimentos de:

\begin{itemize}
    \item Medidas de frequência (absoluta, relativa e acumulada);
    \item Estatística descritiva (média, moda, mediana e desvio padrão);
    \item Tipos de variáveis (qualitativa nominal, qualitativa ordinal, quantitativa discreta e contínua).
\end{itemize}

O formulário deve ser aplicado com, no mínimo, \textbf{20 respondentes reais}. Ao final, o aluno deverá apresentar um relatório com análise dos dados, gráficos e medidas calculadas.

\subsection*{Temas sugeridos}

\begin{enumerate}
    \item \textbf{Sono x Produtividade:} Avaliar padrões de sono e sensação de produtividade. Perguntas podem incluir horas dormidas, horário de dormir, nível de energia ao acordar, percepção de rendimento diário etc.
    
    \item \textbf{Tempo de Estudo x Notas:} Levantar hábitos de estudo e percepção de desempenho acadêmico. Ex: horas de estudo, uso de resumos, revisão antes da prova, notas médias recentes.
    
    \item \textbf{Redes Sociais x Ansiedade:} Investigar tempo de uso das redes e sensações associadas. Pode incluir tipo de rede mais usada, tempo médio por dia, sentimentos após uso, percepção de dependência.
    
    \item \textbf{Estresse x Alimentação:} Avaliar frequência de consumo de fast food, refeições em horários irregulares e níveis de estresse percebidos.
    
    \item \textbf{Atividade Física x Qualidade do Sono:} Levantar frequência de prática de exercícios e qualidade do sono (sono leve/profundo, interrupções, insônia etc.).
    
    \item \textbf{Música x Raciocínio:} Identificar se quem toca instrumentos musicais percebe maior concentração ou desempenho lógico. Perguntas sobre prática musical, tipo de música, tempo de prática, preferências etc.
    
    \item \textbf{Treinos x Autoestima:} Analisar relação entre frequência de exercícios físicos e percepção de autoestima. Incluir perguntas sobre motivação, frequência semanal, tipo de treino, satisfação pessoal etc.
    
    \item \textbf{Cafeína x Produtividade:} Estudar consumo de café/energéticos e impacto no foco. Levantar quantidade de consumo, horário de ingestão, percepção de desempenho antes e após consumir.
    
    \item \textbf{Animais x Produção (Agro):} Coletar dados de pequenas propriedades. Ex: número de vacas, tipo de ração, litros de leite/dia, produtividade percebida.

\hfill
\vspace{1cm}
    
    \item \textbf{Lazer x Felicidade:} Medir frequência de atividades de lazer e percepção de felicidade. Perguntas podem incluir tipo de lazer, frequência, tempo médio, avaliação da semana.
\end{enumerate}

\subsection*{Instruções para entrega do relatório}

\begin{itemize}
    \item Apresentar o link do formulário e um resumo das perguntas aplicadas.
    \item Classificar cada pergunta de acordo com o tipo de variável (qualitativa/quantitativa, discreta/contínua, ordinal/nominal).
    \item Apresentar as tabelas de frequência (absoluta, relativa e acumulada) para ao menos 3 perguntas.
    \item Calcular média, moda e mediana de pelo menos 2 variáveis quantitativas.
    \item Incluir ao menos 2 gráficos (barras ou histograma).
    \item Incluir uma reflexão final com possíveis conclusões ou observações.
    \item O código, conjunto de dados e suas análises deverão estar no \href{https://github.com/claytontey/DS_Unesp/tree/main/Work_Git}{WorkGit} em um diretório com seu nome.
\end{itemize}


Bom trabalho!!!

\end{document}
